This paper investigates the efficacy of Generalized Autoregressive Conditional Heteroskedasticity (GARCH) models in forecasting realized volatility of Bitcoin returns. By employing both rolling-window and expanding-window approaches, we assess the predictive performance of various GARCH-type models against naive benchmarks. Our findings indicate that GARCH-based models significantly outperform naive models. Additionally, we examine the distributional properties of returns and realized volatility, establishing normality, which justifies the choice of normal distribution in GARCH modeling. This research contributes to the understanding of volatility dynamics in cryptocurrency markets, emphasizing the importance of advanced modeling techniques for effective risk management and trading strategies.

